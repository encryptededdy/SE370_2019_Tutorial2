\documentclass{beamer}
%\documentclass[notes=only]{beamer}

\usepackage[utf8]{inputenc}
\usepackage{listings}
\usepackage{ulem}

\lstdefinestyle{customc}{
  belowcaptionskip=1\baselineskip,
  breaklines=true,
%  frame=L,
  xleftmargin=\parindent,
  language=C,
  showstringspaces=false,
  basicstyle=\scriptsize\ttfamily,
  keywordstyle=\bfseries\color{green!40!black},
  commentstyle=\itshape\color{purple!40!black},
  identifierstyle=\color{blue},
  stringstyle=\color{orange},
}

\lstdefinestyle{customctiny}{
  belowcaptionskip=1\baselineskip,
  breaklines=true,
%  frame=L,
  xleftmargin=\parindent,
  language=C,
  showstringspaces=false,
  basicstyle=\tiny\ttfamily,
  keywordstyle=\bfseries\color{green!40!black},
  commentstyle=\itshape\color{purple!40!black},
  identifierstyle=\color{blue},
  stringstyle=\color{orange},
}
 
\usetheme{Dresden}

\title[SOFTENG 370 Tutorial 2 (2019)] %optional
{Forks, pipes and shared memory}
  
\author{Edward Zhang}
 
% \institute[UoA] % (optional)
% {
%   Department of ECSE\\
%   The University of Auckland
% }
 
\date[August 2019] % (optional)
{SOFTENG 370 T2}

\begin{document}
\frame{\titlepage}
\section{Fork}
\begin{frame}
  \frametitle{Fork}
  Fork creates a new process, which becomes a \emph{child} of the caller. Both processes then contiune execution from the line where fork was called.
  \begin{enumerate}
    \item Fork returns negative when forking failed
    \item Fork returns zero within the new child process
    \item Fork returns a positive value with the process id within the parent. This is a \texttt{pid\_t}
  \end{enumerate}
\end{frame}
\begin{frame}
  \frametitle{Fork example}
  \lstinputlisting[style=customc, firstline=5]{code/forkExcercise.c}
  How many times are each line printed, and when is the return zero and non-zero?
\end{frame}
\begin{frame}
  \frametitle{Forked data}
  When forked, all pages allocated for a process are copied. This includes pages that store the stack, or memory on the heap (i.e. from malloc).\\
  Copy-on-Write (CoW) is used so unless the child process modifies the data, a needless copy is not made.

\end{frame}
\begin{frame}
  \frametitle{A note on addresses}
  Consider this code. What's returned?
  \lstinputlisting[style=customc, firstline=5]{code/virtExample.c}
  \pause
  How do we have different data at the same memory address?\\
  \pause
  A: Virtual memory space is unchanged, even if a copy is made in physical memory.
\end{frame}
\end{document}