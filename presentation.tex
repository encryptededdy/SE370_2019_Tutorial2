\documentclass{beamer}
%\documentclass[notes=only]{beamer}

\usepackage[utf8]{inputenc}
\usepackage{listings}
\usepackage{ulem}
\usepackage{amsmath}

\lstdefinestyle{customcbig}{
  belowcaptionskip=1\baselineskip,
  breaklines=true,
%  frame=L,
  xleftmargin=\parindent,
  language=C,
  showstringspaces=false,
  basicstyle=\small\ttfamily,
  keywordstyle=\bfseries\color{green!40!black},
  commentstyle=\itshape\color{purple!40!black},
  identifierstyle=\color{blue},
  stringstyle=\color{orange},
}

\lstdefinestyle{customc}{
  belowcaptionskip=1\baselineskip,
  breaklines=true,
%  frame=L,
  xleftmargin=\parindent,
  language=C,
  showstringspaces=false,
  basicstyle=\scriptsize\ttfamily,
  keywordstyle=\bfseries\color{green!40!black},
  commentstyle=\itshape\color{purple!40!black},
  identifierstyle=\color{blue},
  stringstyle=\color{orange},
}

\lstdefinestyle{customctiny}{
  belowcaptionskip=1\baselineskip,
  breaklines=true,
%  frame=L,
  xleftmargin=\parindent,
  language=C,
  showstringspaces=false,
  basicstyle=\tiny\ttfamily,
  keywordstyle=\bfseries\color{green!40!black},
  commentstyle=\itshape\color{purple!40!black},
  identifierstyle=\color{blue},
  stringstyle=\color{orange},
}
 
\usetheme{Dresden}

\title[SOFTENG 370 Tutorial 2 (2019)] %optional
{Forks, pipes and shared memory}
  
\author{Edward Zhang}
 
% \institute[UoA] % (optional)
% {
%   Department of ECSE\\
%   The University of Auckland
% }
 
\date[August 2019] % (optional)
{SOFTENG 370 T2}

\begin{document}
\frame{\titlepage}
\section{Fork}
\begin{frame}
  \frametitle{Fork}
  Fork creates a new process, which becomes a \emph{child} of the caller. Both processes then contiune execution from the line where fork was called.
  \begin{enumerate}
    \item Fork returns negative when forking failed
    \item Fork returns zero within the new child process
    \item Fork returns a positive value with the process id within the parent. This is a \texttt{pid\_t}
  \end{enumerate}
\end{frame}
\begin{frame}
  \frametitle{Fork example}
  \lstinputlisting[style=customc, firstline=5]{code/forkExcercise.c}
  How many times are each line printed, and when is the return zero and non-zero?
\end{frame}
\begin{frame}
  \frametitle{Forked data}
  When forked, all pages allocated for a process are copied. This includes pages that store the stack, or memory on the heap (i.e. from malloc).\\
  Copy-on-Write (CoW) is used so unless the child process modifies the data, a needless copy is not made.

\end{frame}
\begin{frame}
  \frametitle{A note on addresses}
  Consider this code. What's returned?
  \lstinputlisting[style=customc, firstline=5]{code/virtExample.c}
  \pause
  How do we have different data at the same memory address?\\
  \pause
  A: Virtual memory space is unchanged, even if a copy is made in physical memory.
\end{frame}
\begin{frame}[fragile]
  \frametitle{Waiting for children}
  \texttt{waitpid} can be used to wait for children. Its use is quite simple.
  \begin{lstlisting}[style=customcbig]
    pid_t waitpid(pid_t pid, int *statusPtr, int options);
  \end{lstlisting}
  \begin{itemize}
    \item \texttt{pid} is the pid to wait on. Use -1 for any children, 0 for any child with same group ID or $x < -1$ for children in group $\lvert x\lvert$
    \item \texttt{status} is a pointer to an int to store the return status of the terminated process
    \item \texttt{options} options such as WNOHANG - see manpage
    \item Return value is the id of the terminated process (if you used pid $<1$)
  \end{itemize}
  

\end{frame}
\section{Pipe}
\begin{frame}
  \frametitle{Pipes}
  Consider the case where we have two processes after a fork - how can we communicate between them? We can use a pipe.\\
  \texttt{int pipe(int fds[2])}\\
  \begin{itemize}
    \item \texttt{fd[0]} will be the file descriptor for the read end
    \item \texttt{fd[1]} will be the file descriptor for the write end
    \item 0 returned on success, -1 on error (no other returns)
  \end{itemize}
  We should create this pipe before forking, such that after forking both processes have a reference to the pipe.
\end{frame}
\begin{frame}
  \frametitle{Pipe example}
  Sending strings through a pipe, from the child process to the parent process.
\end{frame}
\begin{frame}
  \frametitle{Other notes}
  \begin{itemize}
    \item Pipes are unidirectional. If you need bidirectional communication use two pipes
    \item Remember to close the side of the pipe you're not using in a given process
    \item \texttt{read} will read 0 bytes when there are no more open write fds on the pipe
    \item While you can have more than 2 processes interact with a pipe, be careful about how you're using it as it may not work the way you intend.
  \end{itemize}
\end{frame}
\section{Shared Memory}
\begin{frame}
  \frametitle{mmap}
  Allows you to map files or devices into memory. However, this can also be used for creating a shared memory mapping that can be used by multiple processes.\newline
  \newline
  You can find it in \texttt{\#include <sys/mman.h>}
\end{frame}
\begin{frame}[fragile]
  \frametitle{mmap parameters}
  \begin{lstlisting}[style=customcbig]
    void *mmap(void *addr, size_t length, int prot, int flags, int fd, off_t offset)
  \end{lstlisting}
  \begin{itemize}
    \item \texttt{addr} is the address we want to start mapping. NULL to let kernel pick
    \item \texttt{length}: size of memory we want to map
    \item \texttt{prot}: options about page (more later)
    \item \texttt{flags}: more page options (more later)
    \item \texttt{fd}: file descriptor, if mapping a file
    \item \texttt{offset}: this plus size is used to compute where we're accessing, in the case of file mapping.
  \end{itemize}
\end{frame}
\begin{frame}[fragile]
  \frametitle{Using mmap for shared memory}
  \begin{lstlisting}[style=customcbig]
    int *shared = mmap(NULL, sizeof(int), PROT_READ|PROT_WRITE, MAP_SHARED|MAP_ANONYMOUS, -1, 0); 
  \end{lstlisting}
  \begin{itemize}
    \item \texttt{addr} is NULL because we don't care where to place the mapping (let kernel pick)
    \item \texttt{length} is the size of an int, as we want to store an int
    \item \texttt{prot} is read/write because... obvious
    \item \texttt{flags} is \texttt{SHARED} (we want shared memory) and \texttt{ANONYMOUS} (as opposed to file mapping)
    \item \texttt{fd} is -1 as we're not mapping a file
    \item \texttt{offset} is 0 as we're not mapping a file
  \end{itemize}
\end{frame}
\begin{frame}
  \frametitle{mmap example code}
  \lstinputlisting[style=customctiny, firstline=5]{code/mmap.c}
\end{frame}
\end{document}